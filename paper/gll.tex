Language predicates and positive/negative labels for those predicates were
gathered through human-robot dialog during the \ispy game.  The human subject
and robot were seated at opposite ends of a small table.  A set of 4 objects
were placed on the table for both to see.  We denote the objects on the table
during a given game as set $O_T$.

\paragraph{Human Turn.} On the subject's turn, the robot asked her or him to
pick an object and describe it in one phrase.  Following~\cite{parde:ijcai15},
we used a standard stopword list to strip out non-content words from the subject's
description.  The remaining words were treated as a set of language predicates,
$H_p$.  To guess which object the subject was describing, the robot assigned
scores to each object on the table.  The score $S$ of each object $o\in O_T$
was calculated as
\begin{equation}
	S(o) = \sum_{p\in H_p}{D_p(o)}
\end{equation}
The robot guessed objects in descending order by score (ties broken randomly)
by pointing at them and asking whether it was correct.  When the correct object
was found, it was added as a positive training example for all classifiers 
$p\in H_p$ used in the subject's description for use in future training.

\paragraph{Robot Turn.} On the robot's turn, an object was chosen at random
from those on the table.  To describe the object, the robot scored the set of
known predicates learned from previous play.  Following Gricean
principles~\cite{grice:bkchapter75}, the robot should describe the object with
predicates that apply to it but do not ambiguously refer to other objects on the
table.  With this in mind, we develop a predicate score that rewards describing
the chosen object $o_c$ and penalizes describing the other objects on the table.
The score $R$ for each predicate $p$ was calculated as
\begin{equation}
	R(p) = |O_T|D_p(o_c) - \sum{o_{d\in{O_T}\setminus\{o_c\}}}{D_p(o_d)}
\end{equation}
Note that this scoring equation also rewards predicates for describing $o_c$
while clearly {\it not} describing the other objects on the table
(\textit{e.g.}  $D_p(o_d)<0$).  The robot choose up to three highest scoring
predicates $\hat{P}$ to describe object $o_c$ to the subject, using fewer if
$S<0$ for any remaining predicates.  Once ready to guess, the subject touched
objects until the robot confirmed that they had guessed the right one
($o_c$).

The robot then pointed to $o_c$ and engaged the user in a brief follow-up
dialog in order to gather both positive and negative labels for $o_c$.  In
addition to predicates $\hat{P}$ used to describe the object, the robot
selected two additional predicates $\bar{P}$ for which it was uncertain whether
the property applied to $o_c$.  $\bar{P}$ were selected randomly with $p\in P$
having a chance of inclusion proportional to $1-|D_p(o_c)|$, such that
classifiers with low confidence in whether or not $p$ applied to $o_c$ were
more likely to be selected.  The robot then asked the subject whether they
would describe the object $o_c$ using each $p\in\hat{P}\cup\bar{P}$.  The
subject's responses to these questions provided additional positive/negative
labels to the classifiers of these predicates for use in future training.
